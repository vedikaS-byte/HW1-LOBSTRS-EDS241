% Options for packages loaded elsewhere
\PassOptionsToPackage{unicode}{hyperref}
\PassOptionsToPackage{hyphens}{url}
%
\documentclass[
]{article}
\usepackage{amsmath,amssymb}
\usepackage{iftex}
\ifPDFTeX
  \usepackage[T1]{fontenc}
  \usepackage[utf8]{inputenc}
  \usepackage{textcomp} % provide euro and other symbols
\else % if luatex or xetex
  \usepackage{unicode-math} % this also loads fontspec
  \defaultfontfeatures{Scale=MatchLowercase}
  \defaultfontfeatures[\rmfamily]{Ligatures=TeX,Scale=1}
\fi
\usepackage{lmodern}
\ifPDFTeX\else
  % xetex/luatex font selection
\fi
% Use upquote if available, for straight quotes in verbatim environments
\IfFileExists{upquote.sty}{\usepackage{upquote}}{}
\IfFileExists{microtype.sty}{% use microtype if available
  \usepackage[]{microtype}
  \UseMicrotypeSet[protrusion]{basicmath} % disable protrusion for tt fonts
}{}
\makeatletter
\@ifundefined{KOMAClassName}{% if non-KOMA class
  \IfFileExists{parskip.sty}{%
    \usepackage{parskip}
  }{% else
    \setlength{\parindent}{0pt}
    \setlength{\parskip}{6pt plus 2pt minus 1pt}}
}{% if KOMA class
  \KOMAoptions{parskip=half}}
\makeatother
\usepackage{xcolor}
\usepackage[margin=1in]{geometry}
\usepackage{color}
\usepackage{fancyvrb}
\newcommand{\VerbBar}{|}
\newcommand{\VERB}{\Verb[commandchars=\\\{\}]}
\DefineVerbatimEnvironment{Highlighting}{Verbatim}{commandchars=\\\{\}}
% Add ',fontsize=\small' for more characters per line
\usepackage{framed}
\definecolor{shadecolor}{RGB}{248,248,248}
\newenvironment{Shaded}{\begin{snugshade}}{\end{snugshade}}
\newcommand{\AlertTok}[1]{\textcolor[rgb]{0.94,0.16,0.16}{#1}}
\newcommand{\AnnotationTok}[1]{\textcolor[rgb]{0.56,0.35,0.01}{\textbf{\textit{#1}}}}
\newcommand{\AttributeTok}[1]{\textcolor[rgb]{0.13,0.29,0.53}{#1}}
\newcommand{\BaseNTok}[1]{\textcolor[rgb]{0.00,0.00,0.81}{#1}}
\newcommand{\BuiltInTok}[1]{#1}
\newcommand{\CharTok}[1]{\textcolor[rgb]{0.31,0.60,0.02}{#1}}
\newcommand{\CommentTok}[1]{\textcolor[rgb]{0.56,0.35,0.01}{\textit{#1}}}
\newcommand{\CommentVarTok}[1]{\textcolor[rgb]{0.56,0.35,0.01}{\textbf{\textit{#1}}}}
\newcommand{\ConstantTok}[1]{\textcolor[rgb]{0.56,0.35,0.01}{#1}}
\newcommand{\ControlFlowTok}[1]{\textcolor[rgb]{0.13,0.29,0.53}{\textbf{#1}}}
\newcommand{\DataTypeTok}[1]{\textcolor[rgb]{0.13,0.29,0.53}{#1}}
\newcommand{\DecValTok}[1]{\textcolor[rgb]{0.00,0.00,0.81}{#1}}
\newcommand{\DocumentationTok}[1]{\textcolor[rgb]{0.56,0.35,0.01}{\textbf{\textit{#1}}}}
\newcommand{\ErrorTok}[1]{\textcolor[rgb]{0.64,0.00,0.00}{\textbf{#1}}}
\newcommand{\ExtensionTok}[1]{#1}
\newcommand{\FloatTok}[1]{\textcolor[rgb]{0.00,0.00,0.81}{#1}}
\newcommand{\FunctionTok}[1]{\textcolor[rgb]{0.13,0.29,0.53}{\textbf{#1}}}
\newcommand{\ImportTok}[1]{#1}
\newcommand{\InformationTok}[1]{\textcolor[rgb]{0.56,0.35,0.01}{\textbf{\textit{#1}}}}
\newcommand{\KeywordTok}[1]{\textcolor[rgb]{0.13,0.29,0.53}{\textbf{#1}}}
\newcommand{\NormalTok}[1]{#1}
\newcommand{\OperatorTok}[1]{\textcolor[rgb]{0.81,0.36,0.00}{\textbf{#1}}}
\newcommand{\OtherTok}[1]{\textcolor[rgb]{0.56,0.35,0.01}{#1}}
\newcommand{\PreprocessorTok}[1]{\textcolor[rgb]{0.56,0.35,0.01}{\textit{#1}}}
\newcommand{\RegionMarkerTok}[1]{#1}
\newcommand{\SpecialCharTok}[1]{\textcolor[rgb]{0.81,0.36,0.00}{\textbf{#1}}}
\newcommand{\SpecialStringTok}[1]{\textcolor[rgb]{0.31,0.60,0.02}{#1}}
\newcommand{\StringTok}[1]{\textcolor[rgb]{0.31,0.60,0.02}{#1}}
\newcommand{\VariableTok}[1]{\textcolor[rgb]{0.00,0.00,0.00}{#1}}
\newcommand{\VerbatimStringTok}[1]{\textcolor[rgb]{0.31,0.60,0.02}{#1}}
\newcommand{\WarningTok}[1]{\textcolor[rgb]{0.56,0.35,0.01}{\textbf{\textit{#1}}}}
\usepackage{graphicx}
\makeatletter
\newsavebox\pandoc@box
\newcommand*\pandocbounded[1]{% scales image to fit in text height/width
  \sbox\pandoc@box{#1}%
  \Gscale@div\@tempa{\textheight}{\dimexpr\ht\pandoc@box+\dp\pandoc@box\relax}%
  \Gscale@div\@tempb{\linewidth}{\wd\pandoc@box}%
  \ifdim\@tempb\p@<\@tempa\p@\let\@tempa\@tempb\fi% select the smaller of both
  \ifdim\@tempa\p@<\p@\scalebox{\@tempa}{\usebox\pandoc@box}%
  \else\usebox{\pandoc@box}%
  \fi%
}
% Set default figure placement to htbp
\def\fps@figure{htbp}
\makeatother
\setlength{\emergencystretch}{3em} % prevent overfull lines
\providecommand{\tightlist}{%
  \setlength{\itemsep}{0pt}\setlength{\parskip}{0pt}}
\setcounter{secnumdepth}{-\maxdimen} % remove section numbering
\usepackage{booktabs}
\usepackage{caption}
\usepackage{longtable}
\usepackage{colortbl}
\usepackage{array}
\usepackage{anyfontsize}
\usepackage{multirow}
\usepackage{wrapfig}
\usepackage{float}
\usepackage{pdflscape}
\usepackage{tabu}
\usepackage{threeparttable}
\usepackage{threeparttablex}
\usepackage[normalem]{ulem}
\usepackage{makecell}
\usepackage{xcolor}
\usepackage{graphicx}
\usepackage{siunitx}
\usepackage{hhline}
\usepackage{calc}
\usepackage{tabularx}
\usepackage{adjustbox}
\usepackage{hyperref}
\usepackage{bookmark}
\IfFileExists{xurl.sty}{\usepackage{xurl}}{} % add URL line breaks if available
\urlstyle{same}
\hypersetup{
  pdftitle={Assignment 1},
  pdfauthor={EDS 241 / ESM 244 (Due: 1/20)},
  hidelinks,
  pdfcreator={LaTeX via pandoc}}

\title{Assignment 1}
\usepackage{etoolbox}
\makeatletter
\providecommand{\subtitle}[1]{% add subtitle to \maketitle
  \apptocmd{\@title}{\par {\large #1 \par}}{}{}
}
\makeatother
\subtitle{California Spiny Lobster (\emph{Panulirus interruptus}):
Assessing the Impact of Marine Protected Areas (MPAs) at 5 Reef Sites in
Santa Barbara County}
\author{EDS 241 / ESM 244 (\textbf{Due: 1/20})}
\date{1/8/26}

\begin{document}
\maketitle

\begin{center}\rule{0.5\linewidth}{0.5pt}\end{center}

\pandocbounded{\includegraphics[keepaspectratio]{figures/spiny2.jpg}}

\begin{center}\rule{0.5\linewidth}{0.5pt}\end{center}

\subsubsection{Assignment Instructions:}\label{assignment-instructions}

\begin{itemize}
\item
  Working with partners to troubleshoot code and concepts is encouraged!
  If you work with a partner, please list their name next to yours at
  the top of your assignment so Annie and I can easily see who
  collaborated.
\item
  All written responses must be written independently (\textbf{in your
  own words}).
\item
  Please follow the question prompts carefully and include only the
  information each question asks in your submitted responses.
\item
  Submit both your knitted document and the associated
  \texttt{RMarkdown} or \texttt{Quarto} file.
\item
  Your knitted presentation should meet the quality you'd submit to
  research colleagues or feel confident sharing publicly. Refer to the
  rubric for details about presentation standards.
\end{itemize}

\textbf{Assignment submission (YOUR NAME):} Vedika Shirtekar

\begin{center}\rule{0.5\linewidth}{0.5pt}\end{center}

\begin{Shaded}
\begin{Highlighting}[]
\FunctionTok{library}\NormalTok{(tidyverse)}
\FunctionTok{library}\NormalTok{(here)}
\FunctionTok{library}\NormalTok{(janitor)}
\FunctionTok{library}\NormalTok{(estimatr)  }
\FunctionTok{library}\NormalTok{(performance)}
\FunctionTok{library}\NormalTok{(jtools)}
\FunctionTok{library}\NormalTok{(gt)}
\FunctionTok{library}\NormalTok{(gtsummary)}
\FunctionTok{library}\NormalTok{(interactions) }
\FunctionTok{library}\NormalTok{(ggridges)}
\end{Highlighting}
\end{Shaded}

\begin{center}\rule{0.5\linewidth}{0.5pt}\end{center}

\paragraph{DATA SOURCE:}\label{data-source}

\begin{quote}
\href{https://doi.org/10.6073/pasta/a593a675d644fdefb736750b291579a0}{Reed
D. 2019. SBC LTER: Reef: Abundance, size and fishing effort for
California Spiny Lobster (Panulirus interruptus), ongoing since 2012.
Environmental Data Initiative.} Data accessed 11/17/2019.
\end{quote}

\begin{center}\rule{0.5\linewidth}{0.5pt}\end{center}

\subsubsection{\texorpdfstring{\textbf{Introduction}}{Introduction}}\label{introduction}

You're about to dive into some deep data collected from five reef sites
in Santa Barbara County, all about the abundance of California spiny
lobsters! Data was gathered by divers annually from 2012 to 2018 across
Naples, Mohawk, Isla Vista, Carpinteria, and Arroyo Quemado reefs.

Why lobsters? Well, this sample provides an opportunity to evaluate the
impact of Marine Protected Areas (MPAs) established on January 1, 2012
(Reed, 2019). Of these five reefs, Naples, and Isla Vista are MPAs,
while the other three are not protected (non-MPAs). Comparing lobster
health between these protected and non-protected areas gives us the
chance to study how commercial and recreational fishing might impact
these ecosystems.

We will consider the MPA sites the \texttt{treatment} group and use
regression methods to explore whether protecting these reefs really
makes a difference compared to non-MPA sites (our control group). In
this assignment, we'll think deeply about which causal inference
assumptions hold up under the research design and identify where they
fall short.

Let's break it down step by step and see what the data reveals!

\pandocbounded{\includegraphics[keepaspectratio]{figures/map-5reefs.png}}

\begin{center}\rule{0.5\linewidth}{0.5pt}\end{center}

\paragraph{Step 1: Anticipating potential sources of selection
bias}\label{step-1-anticipating-potential-sources-of-selection-bias}

\textbf{a.} Do the control sites (Arroyo Quemado, Carpenteria, and
Mohawk) provide a strong counterfactual for our treatment sites (Naples,
Isla Vista)? Write a paragraph making a case for why this comparison is
ceteris paribus or whether selection bias is likely (be specific!).

The control sites (Arroyo Quemado, Carpenteria, and Mohawk) are
reasonably located along the same coastline as the treatment sites
(Naples and Isla Vista), helping to control for larger-scale environment
differences such as water temperature, wave intensity, and additional
environmental factors. However, there are potential sources of selection
bias, as the treatment sites were chosen for MPA protection and could
potentially be correlated with preexisting ecological conditions (ex.
high lobster densities or improved habitat quality) that differ from the
control sites (non-MPAs). While the proximity of the control sites can
provide the \emph{ceteris paribus} comparison for regional environmental
conditions, influence from human activity or differences in reef
structure could influence spiny lobster counts independent of MPA
status.

According to the metadata, sites (reefs) were not randomly assigned to
treatment and control conditions. Instead, the two treatment sites
(Naples and Isla Vista) are reefs that already happened to fall within
California's MPA network when the MPAs were established in 2012. The
remaining sites are long term kelp forest monitoring reefs located
outside MPAs and serve as comparison (or control) sites. As a result,
assignment to MPA versus non-MPA status is observational and not the
result of an experimental randomization process.

Step 2: Read \& wrangle data

\textbf{a.} Read in the raw data from the ``data'' folder named
\texttt{spiny\_abundance\_sb\_18.csv}. Name the data.frame
\texttt{rawdata}

\textbf{b.} Use the function \texttt{clean\_names()} from the
\texttt{janitor} package

\begin{Shaded}
\begin{Highlighting}[]
\CommentTok{\# HINT: check for coding of missing values (\textasciigrave{}na = "{-}99999"\textasciigrave{})}
\NormalTok{rawdata }\OtherTok{\textless{}{-}} 
    \FunctionTok{read\_csv}\NormalTok{(here}\SpecialCharTok{::}\FunctionTok{here}\NormalTok{(}\StringTok{"data"}\NormalTok{, }\StringTok{"spiny\_abundance\_sb\_18.csv"}\NormalTok{), }\AttributeTok{na =} \StringTok{"{-}99999"}\NormalTok{) }\SpecialCharTok{\%\textgreater{}\%} 
    \FunctionTok{clean\_names}\NormalTok{()}
\end{Highlighting}
\end{Shaded}

\textbf{c.} Create a new \texttt{df} named \texttt{tidyata}. Using the
variable \texttt{site} (reef location) create a new variable
\texttt{reef} as a \texttt{factor} and add the following labels in the
order listed (i.e., re-order the \texttt{levels}):

\begin{verbatim}
"Arroyo Quemado", "Carpenteria", "Mohawk", "Isla Vista",  "Naples"
\end{verbatim}

\begin{Shaded}
\begin{Highlighting}[]
\NormalTok{tidydata }\OtherTok{\textless{}{-}}\NormalTok{ rawdata }\SpecialCharTok{\%\textgreater{}\%} 
  \FunctionTok{mutate}\NormalTok{(}
    \AttributeTok{reef =} \FunctionTok{factor}\NormalTok{(}
      \FunctionTok{case\_when}\NormalTok{(}
\NormalTok{        site }\SpecialCharTok{==} \StringTok{"AQUE"} \SpecialCharTok{\textasciitilde{}} \StringTok{"Arroyo Quemado"}\NormalTok{, }
\NormalTok{        site }\SpecialCharTok{==} \StringTok{"CARP"} \SpecialCharTok{\textasciitilde{}} \StringTok{"Carpenteria"}\NormalTok{, }
\NormalTok{        site }\SpecialCharTok{==} \StringTok{"MOHK"} \SpecialCharTok{\textasciitilde{}} \StringTok{"Mohawk"}\NormalTok{, }
\NormalTok{        site }\SpecialCharTok{==} \StringTok{"IVEE"} \SpecialCharTok{\textasciitilde{}} \StringTok{"Isla Vista"}\NormalTok{, }
\NormalTok{        site }\SpecialCharTok{==} \StringTok{"NAPL"} \SpecialCharTok{\textasciitilde{}} \StringTok{"Naples"}
\NormalTok{      ),}
      \AttributeTok{levels =} \FunctionTok{c}\NormalTok{(}
        \StringTok{"Arroyo Quemado"}\NormalTok{,}
        \StringTok{"Carpenteria"}\NormalTok{,}
        \StringTok{"Mohawk"}\NormalTok{,}
        \StringTok{"Isla Vista"}\NormalTok{,}
        \StringTok{"Naples"}
\NormalTok{      )}
\NormalTok{    )}
\NormalTok{  )}
\end{Highlighting}
\end{Shaded}

Create new \texttt{df} named \texttt{spiny\_counts}

\textbf{d.} Create a new variable \texttt{counts} to allow for an
analysis of lobster counts where the unit-level of observation is the
total number of observed lobsters per \texttt{site}, \texttt{year} and
\texttt{transect}.

\begin{itemize}
\tightlist
\item
  Create a variable \texttt{mean\_size} from the variable
  \texttt{size\_mm}
\item
  NOTE: The variable \texttt{counts} should have values which are
  integers (whole numbers).
\item
  Make sure to account for missing cases (\texttt{na})!
\end{itemize}

\textbf{e.} Create a new variable \texttt{mpa} with levels \texttt{MPA}
and \texttt{non\_MPA}. For our regression analysis create a numerical
variable \texttt{treat} where MPA sites are coded \texttt{1} and
non\_MPA sites are coded \texttt{0}

\begin{Shaded}
\begin{Highlighting}[]
\CommentTok{\#HINT(d): Use \textasciigrave{}group\_by()\textasciigrave{} \& \textasciigrave{}summarize()\textasciigrave{} to provide the total number of lobsters observed at each site{-}year{-}transect row{-}observation. }
\NormalTok{counts }\OtherTok{\textless{}{-}}\NormalTok{ tidydata }\SpecialCharTok{\%\textgreater{}\%} 
 \CommentTok{\# mutate(size\_mm = na\_if(size\_mm, {-}99999)) \%\textgreater{}\% }
  \FunctionTok{group\_by}\NormalTok{(site, year, transect, reef) }\SpecialCharTok{\%\textgreater{}\%} 
  \FunctionTok{summarize}\NormalTok{(}
    \AttributeTok{count =} \FunctionTok{sum}\NormalTok{(count, }\AttributeTok{na.rm =} \ConstantTok{TRUE}\NormalTok{),}
    \AttributeTok{mean\_size =} \FunctionTok{mean}\NormalTok{(size\_mm, }\AttributeTok{na.rm =} \ConstantTok{TRUE}\NormalTok{),}
    \AttributeTok{.groups =} \StringTok{"drop"}
\NormalTok{  )}

\CommentTok{\#HINT(e): Use \textasciigrave{}case\_when()\textasciigrave{} to create the 3 new variable columns}
\NormalTok{spiny\_counts }\OtherTok{\textless{}{-}}\NormalTok{ counts }\SpecialCharTok{\%\textgreater{}\%} 
  \FunctionTok{mutate}\NormalTok{( }
    \AttributeTok{mpa =} 
      \FunctionTok{case\_when}\NormalTok{(}
\NormalTok{        site }\SpecialCharTok{\%in\%} \FunctionTok{c}\NormalTok{(}\StringTok{"NAPL"}\NormalTok{, }\StringTok{"IVEE"}\NormalTok{) }\SpecialCharTok{\textasciitilde{}} \StringTok{"MPA"}\NormalTok{,}
\NormalTok{        site }\SpecialCharTok{\%in\%} \FunctionTok{c}\NormalTok{(}\StringTok{"AQUE"}\NormalTok{, }\StringTok{"CARP"}\NormalTok{, }\StringTok{"MOHK"}\NormalTok{) }\SpecialCharTok{\textasciitilde{}} \StringTok{"non\_MPA"}\NormalTok{), }
    \AttributeTok{treat =} 
      \FunctionTok{case\_when}\NormalTok{(}
\NormalTok{        mpa }\SpecialCharTok{==} \StringTok{"MPA"} \SpecialCharTok{\textasciitilde{}} \DecValTok{1}\NormalTok{, }
\NormalTok{        mpa }\SpecialCharTok{==} \StringTok{"non\_MPA"} \SpecialCharTok{\textasciitilde{}} \DecValTok{0}
        
\NormalTok{      )}
\NormalTok{    ) }
\end{Highlighting}
\end{Shaded}

\begin{quote}
NOTE: This step is crucial to the analysis. Check with a friend or come
to TA/instructor office hours to make sure the counts are coded
correctly!
\end{quote}

\begin{center}\rule{0.5\linewidth}{0.5pt}\end{center}

\paragraph{Step 3: Explore \& visualize
data}\label{step-3-explore-visualize-data}

\textbf{a.} Take a look at the data! Get familiar with the data in each
\texttt{df} format (\texttt{tidydata}, \texttt{spiny\_counts})

\begin{Shaded}
\begin{Highlighting}[]
\DocumentationTok{\#\# tidydata}
\FunctionTok{str}\NormalTok{(tidydata)}
\end{Highlighting}
\end{Shaded}

\begin{verbatim}
## tibble [4,362 x 11] (S3: tbl_df/tbl/data.frame)
##  $ year     : num [1:4362] 2012 2012 2012 2012 2012 ...
##  $ month    : num [1:4362] 8 8 8 8 8 8 8 8 8 8 ...
##  $ date     : Date[1:4362], format: "2012-08-20" "2012-08-20" ...
##  $ site     : chr [1:4362] "IVEE" "IVEE" "IVEE" "IVEE" ...
##  $ transect : num [1:4362] 1 1 1 1 2 2 2 2 3 3 ...
##  $ replicate: chr [1:4362] "A" "B" "C" "D" ...
##  $ size_mm  : num [1:4362] NA NA NA NA NA NA NA NA 70 60 ...
##  $ count    : num [1:4362] 0 0 0 0 0 0 0 0 1 1 ...
##  $ num_ao   : num [1:4362] 0 0 0 0 0 0 0 0 0 0 ...
##  $ area     : num [1:4362] 300 300 300 300 300 300 300 300 300 300 ...
##  $ reef     : Factor w/ 5 levels "Arroyo Quemado",..: 4 4 4 4 4 4 4 4 4 4 ...
\end{verbatim}

\begin{Shaded}
\begin{Highlighting}[]
\DocumentationTok{\#\# spiny\_counts}
\FunctionTok{str}\NormalTok{(spiny\_counts)}
\end{Highlighting}
\end{Shaded}

\begin{verbatim}
## tibble [252 x 8] (S3: tbl_df/tbl/data.frame)
##  $ site     : chr [1:252] "AQUE" "AQUE" "AQUE" "AQUE" ...
##  $ year     : num [1:252] 2012 2012 2012 2012 2012 ...
##  $ transect : num [1:252] 1 2 3 4 5 6 7 1 2 3 ...
##  $ reef     : Factor w/ 5 levels "Arroyo Quemado",..: 1 1 1 1 1 1 1 1 1 1 ...
##  $ count    : num [1:252] 5 9 0 9 11 0 4 7 2 0 ...
##  $ mean_size: num [1:252] 64.2 66 NaN 74.1 76.9 ...
##  $ mpa      : chr [1:252] "non_MPA" "non_MPA" "non_MPA" "non_MPA" ...
##  $ treat    : num [1:252] 0 0 0 0 0 0 0 0 0 0 ...
\end{verbatim}

\textbf{b.} We will focus on the variables \texttt{count},
\texttt{year}, \texttt{site}, and \texttt{treat}(\texttt{mpa}) to model
lobster abundance. Create the following 4 plots using a different method
each time from the 6 options provided. Add a layer (\texttt{geom}) to
each of the plots including informative descriptive statistics (you
choose; e.g., mean, median, SD, quartiles, range). Make sure each plot
dimension is clearly labeled (e.g., axes, groups).

\begin{itemize}
\tightlist
\item
  \href{https://r-charts.com/distribution/density-plot-group-ggplot2}{Density
  plot}
\item
  \href{https://r-charts.com/distribution/ggridges/}{Ridge plot}
\item
  \href{https://ggplot2.tidyverse.org/reference/geom_jitter.html}{Jitter
  plot}
\item
  \href{https://r-charts.com/distribution/violin-plot-group-ggplot2}{Violin
  plot}
\item
  \href{https://r-charts.com/distribution/histogram-density-ggplot2/}{Histogram}
\item
  \href{https://r-charts.com/distribution/beeswarm/}{Beeswarm}
\end{itemize}

Create plots displaying the distribution of lobster \textbf{counts}:

\begin{enumerate}
\def\labelenumi{\arabic{enumi})}
\tightlist
\item
  grouped by reef site
\item
  grouped by MPA status
\item
  grouped by year
\end{enumerate}

Create a plot of lobster \textbf{size} :

\begin{enumerate}
\def\labelenumi{\arabic{enumi})}
\setcounter{enumi}{3}
\tightlist
\item
  You choose the grouping variable(s)!
\end{enumerate}

\begin{Shaded}
\begin{Highlighting}[]
\CommentTok{\# plot 1: Distribution of Spiny Lobster Counts by Reef Site}
\NormalTok{site\_medians }\OtherTok{\textless{}{-}}\NormalTok{ spiny\_counts }\SpecialCharTok{\%\textgreater{}\%}
  \FunctionTok{group\_by}\NormalTok{(site) }\SpecialCharTok{\%\textgreater{}\%} 
  \FunctionTok{summarize}\NormalTok{(}\AttributeTok{median\_count =} \FunctionTok{median}\NormalTok{(count), }\AttributeTok{.groups =} \StringTok{"drop"}\NormalTok{)}


\NormalTok{spiny\_counts }\SpecialCharTok{\%\textgreater{}\%}
  \FunctionTok{ggplot}\NormalTok{(}\FunctionTok{aes}\NormalTok{(}\AttributeTok{x =}\NormalTok{ count, }\AttributeTok{fill =}\NormalTok{ site)) }\SpecialCharTok{+}
  \FunctionTok{geom\_histogram}\NormalTok{(}\AttributeTok{bins =} \DecValTok{35}\NormalTok{, }\AttributeTok{color =} \StringTok{"white"}\NormalTok{, }\AttributeTok{alpha =} \FloatTok{0.6}\NormalTok{) }\SpecialCharTok{+}
  \FunctionTok{geom\_vline}\NormalTok{(}
    \AttributeTok{data =}\NormalTok{ site\_medians,}
    \FunctionTok{aes}\NormalTok{(}\AttributeTok{xintercept =}\NormalTok{ median\_count),}
    \AttributeTok{color =} \StringTok{"darkblue"}\NormalTok{,}
    \AttributeTok{linetype =} \StringTok{"dashed"}\NormalTok{,}
    \AttributeTok{linewidth =}\NormalTok{ .}\DecValTok{5} 
\NormalTok{  ) }\SpecialCharTok{+}
    \FunctionTok{scale\_fill\_brewer}\NormalTok{(}\AttributeTok{palette =} \StringTok{"Set1"}\NormalTok{) }\SpecialCharTok{+} 

  \FunctionTok{facet\_wrap}\NormalTok{(}\SpecialCharTok{\textasciitilde{}}\NormalTok{site, }\AttributeTok{scales =} \StringTok{"free\_y"}\NormalTok{) }\SpecialCharTok{+}
  \FunctionTok{theme\_bw}\NormalTok{(}\AttributeTok{base\_size =} \DecValTok{10}\NormalTok{) }\SpecialCharTok{+}
  \FunctionTok{theme}\NormalTok{(}\AttributeTok{legend.position =} \StringTok{"none"}\NormalTok{) }\SpecialCharTok{+}
  \FunctionTok{labs}\NormalTok{(}
    \AttributeTok{x =} \StringTok{"Lobster Counts"}\NormalTok{,}
    \AttributeTok{y =} \StringTok{"Frequency"}\NormalTok{,}
    \AttributeTok{title =} \StringTok{"Distribution of Spiny Lobster Counts by Reef Site"}\NormalTok{,}
    \AttributeTok{subtitle =} \StringTok{"Dashed line indicates median lobster abundance per site"}
\NormalTok{  )}
\end{Highlighting}
\end{Shaded}

\pandocbounded{\includegraphics[keepaspectratio]{hw1-lobstrs-eds241_files/figure-latex/unnamed-chunk-6-1.pdf}}

\begin{Shaded}
\begin{Highlighting}[]
\CommentTok{\# plot 2.. Distribution of Spiny Lobster Counts by Treatment (MPA Designation)}
\NormalTok{spiny\_counts }\SpecialCharTok{\%\textgreater{}\%}
    \FunctionTok{mutate}\NormalTok{(}\AttributeTok{treat =} \FunctionTok{factor}\NormalTok{(treat, }\AttributeTok{levels =} \FunctionTok{c}\NormalTok{(}\DecValTok{0}\NormalTok{, }\DecValTok{1}\NormalTok{), }\AttributeTok{labels =} \FunctionTok{c}\NormalTok{(}\StringTok{"Non{-}MPA"}\NormalTok{, }\StringTok{"MPA"}\NormalTok{))) }\SpecialCharTok{\%\textgreater{}\%}
  \FunctionTok{ggplot}\NormalTok{(}\FunctionTok{aes}\NormalTok{(}\AttributeTok{x =} \FunctionTok{factor}\NormalTok{(treat), }\AttributeTok{y =}\NormalTok{ count, }\AttributeTok{fill =}\NormalTok{ treat)) }\SpecialCharTok{+}
  \FunctionTok{geom\_violin}\NormalTok{(}\AttributeTok{trim =} \ConstantTok{FALSE}\NormalTok{, }\AttributeTok{draw\_quantiles =} \FunctionTok{c}\NormalTok{(}\FloatTok{0.25}\NormalTok{, }\FloatTok{0.5}\NormalTok{, }\FloatTok{0.75}\NormalTok{), }\AttributeTok{alpha =} \FloatTok{0.7}\NormalTok{) }\SpecialCharTok{+}
  \FunctionTok{geom\_jitter}\NormalTok{(}\AttributeTok{width =} \FloatTok{0.15}\NormalTok{, }\AttributeTok{size =} \DecValTok{1}\NormalTok{, }\AttributeTok{alpha =} \FloatTok{0.5}\NormalTok{, }\AttributeTok{color =} \StringTok{"black"}\NormalTok{) }\SpecialCharTok{+}
  \FunctionTok{scale\_fill\_manual}\NormalTok{(}\AttributeTok{values =} \FunctionTok{c}\NormalTok{(}\StringTok{"Non{-}MPA"} \OtherTok{=} \StringTok{"\#FB6A4A"}\NormalTok{, }\StringTok{"MPA"} \OtherTok{=} \StringTok{"\#6BAED6"}\NormalTok{)) }\SpecialCharTok{+}
  \FunctionTok{theme\_bw}\NormalTok{(}\AttributeTok{base\_size =} \DecValTok{10}\NormalTok{) }\SpecialCharTok{+} \FunctionTok{guides}\NormalTok{(}\AttributeTok{fill =} \StringTok{"none"}\NormalTok{) }\SpecialCharTok{+}
  \FunctionTok{labs}\NormalTok{(}
    \AttributeTok{x =} \StringTok{"Treatment"}\NormalTok{,}
    \AttributeTok{y =} \StringTok{"Lobster Counts"}\NormalTok{,}
    \AttributeTok{fill =} \StringTok{"MPA Designation"}\NormalTok{,}
    \AttributeTok{title =} \StringTok{"Distribution of Spiny Lobster Counts by Treatment (MPA Designation)"}\NormalTok{,}
    \AttributeTok{subtitle =} \StringTok{"Lines reference quartiles (25\%, 50\%, 75\%)"}
\NormalTok{  )}
\end{Highlighting}
\end{Shaded}

\pandocbounded{\includegraphics[keepaspectratio]{hw1-lobstrs-eds241_files/figure-latex/unnamed-chunk-7-1.pdf}}

\begin{Shaded}
\begin{Highlighting}[]
\CommentTok{\# plot 3... Distribution of Spiny Lobster Counts (2012{-}2018)}
\NormalTok{ spiny\_counts }\SpecialCharTok{\%\textgreater{}\%}
  \FunctionTok{ggplot}\NormalTok{(}
    \FunctionTok{aes}\NormalTok{(}
      \AttributeTok{x =}\NormalTok{ count,}
      \AttributeTok{y =} \FunctionTok{fct\_rev}\NormalTok{(}\FunctionTok{factor}\NormalTok{(year)),}
      \AttributeTok{fill =} \FunctionTok{stat}\NormalTok{(quantile)}
\NormalTok{    )}
\NormalTok{  ) }\SpecialCharTok{+}
  \FunctionTok{geom\_density\_ridges\_gradient}\NormalTok{(}
    \AttributeTok{quantiles =} \FunctionTok{c}\NormalTok{(}\FloatTok{0.25}\NormalTok{, }\FloatTok{0.5}\NormalTok{, }\FloatTok{0.75}\NormalTok{),}
    \AttributeTok{quantile\_lines =}\NormalTok{ T,}
    \AttributeTok{scale =} \DecValTok{1}\NormalTok{,}
    \AttributeTok{alpha =} \FloatTok{0.3}
\NormalTok{  ) }\SpecialCharTok{+} \FunctionTok{scale\_fill\_viridis\_d}\NormalTok{(}\AttributeTok{option =} \StringTok{"J"}\NormalTok{, }\AttributeTok{alpha =}\NormalTok{ .}\DecValTok{6}\NormalTok{)}\SpecialCharTok{+} 
  \FunctionTok{theme\_bw}\NormalTok{(}\AttributeTok{base\_size =} \DecValTok{10}\NormalTok{) }\SpecialCharTok{+}
  \FunctionTok{labs}\NormalTok{(}
    \AttributeTok{title =} \StringTok{"Distribution of Spiny Lobster Counts (2012{-}2018)"}\NormalTok{,}
    \AttributeTok{subtitle =} \StringTok{"IQR (25{-}75\%) outlined"}\NormalTok{,}
    \AttributeTok{x =} \StringTok{"Lobster Counts"}\NormalTok{,}
    \AttributeTok{y =} \StringTok{"Year"}\NormalTok{, }
    \AttributeTok{fill =} \StringTok{"Quantile"}
\NormalTok{  ) }\SpecialCharTok{+} 
    \FunctionTok{coord\_cartesian}\NormalTok{(}\AttributeTok{xlim =} \FunctionTok{c}\NormalTok{(}\ConstantTok{NA}\NormalTok{, }\DecValTok{120}\NormalTok{))}
\end{Highlighting}
\end{Shaded}

\pandocbounded{\includegraphics[keepaspectratio]{hw1-lobstrs-eds241_files/figure-latex/unnamed-chunk-8-1.pdf}}

\begin{Shaded}
\begin{Highlighting}[]
\CommentTok{\# plot 4... Mean lobster size by reef with highlighted average of reef mean sizes}
\NormalTok{beeswarm}\SpecialCharTok{::}\FunctionTok{beeswarm}\NormalTok{(}
\NormalTok{  mean\_size }\SpecialCharTok{\textasciitilde{}} \FunctionTok{factor}\NormalTok{(reef), }
  \AttributeTok{pch =} \DecValTok{19}\NormalTok{, }
  \AttributeTok{data =}\NormalTok{ spiny\_counts,}
  \AttributeTok{col =} \DecValTok{2}\SpecialCharTok{:}\DecValTok{6}\NormalTok{, }
  \AttributeTok{cex.axis =} \FloatTok{0.8}\NormalTok{,}
  \AttributeTok{main =} \StringTok{"Distribution of Mean Lobster Sizes by Reef }\SpecialCharTok{\textbackslash{}n}\StringTok{(Diamonds represent reef{-}level means)"}\NormalTok{,}
  \AttributeTok{xlab =} \StringTok{"Reef"}\NormalTok{,}
  \AttributeTok{ylab =} \StringTok{"Mean Lobster Size (mm)"}
\NormalTok{)}

\NormalTok{reef\_means }\OtherTok{\textless{}{-}}\NormalTok{ spiny\_counts }\SpecialCharTok{\%\textgreater{}\%} 
  \FunctionTok{group\_by}\NormalTok{(reef) }\SpecialCharTok{\%\textgreater{}\%} 
  \FunctionTok{summarize}\NormalTok{(}\AttributeTok{average\_mean\_dist =} \FunctionTok{mean}\NormalTok{(mean\_size, }\AttributeTok{na.rm =} \ConstantTok{TRUE}\NormalTok{))}

\CommentTok{\# Add on point to indicate reef{-}level mean (contrary to ggplot, need to add manually)}
\FunctionTok{points}\NormalTok{( }\CommentTok{\# Convert to numeric because points() expects numeric coordinates}
  \AttributeTok{x =} \FunctionTok{as.numeric}\NormalTok{(}\FunctionTok{factor}\NormalTok{(reef\_means}\SpecialCharTok{$}\NormalTok{reef)),}
  \AttributeTok{y =}\NormalTok{ reef\_means}\SpecialCharTok{$}\NormalTok{average\_mean\_dist,}
  \AttributeTok{pch =} \DecValTok{23}\NormalTok{,  }
  \AttributeTok{bg =} \StringTok{"black"}
\NormalTok{) }
\end{Highlighting}
\end{Shaded}

\pandocbounded{\includegraphics[keepaspectratio]{hw1-lobstrs-eds241_files/figure-latex/unnamed-chunk-9-1.pdf}}

\textbf{c.} Compare means of the outcome by treatment group. Using the
\texttt{tbl\_summary()} function from the package
\href{https://www.danieldsjoberg.com/gtsummary/articles/tbl_summary.html}{\texttt{gt\_summary}}

\begin{Shaded}
\begin{Highlighting}[]
\CommentTok{\# USE: gt\_summary::tbl\_summary()}
\NormalTok{spiny\_counts }\SpecialCharTok{\%\textgreater{}\%} 
    \FunctionTok{tbl\_summary}\NormalTok{(}\AttributeTok{by =}\NormalTok{ mpa, }\AttributeTok{include =}\NormalTok{ count, }
                             \AttributeTok{statistic =} \FunctionTok{list}\NormalTok{(}\FunctionTok{all\_continuous}\NormalTok{() }\SpecialCharTok{\textasciitilde{}} \StringTok{"\{mean\} (\{sd\})"}\NormalTok{),}
                             \AttributeTok{missing =} \StringTok{"no"}\NormalTok{)  }\SpecialCharTok{\%\textgreater{}\%} \FunctionTok{add\_p}\NormalTok{()}
\end{Highlighting}
\end{Shaded}

\begin{table}[t]
\fontsize{12.0pt}{14.0pt}\selectfont
\begin{tabular*}{\linewidth}{@{\extracolsep{\fill}}lccc}
\toprule
\textbf{Characteristic} & \textbf{MPA}  N = 119\textsuperscript{\textit{1}} & \textbf{non\_MPA}  N = 133\textsuperscript{\textit{1}} & \textbf{p-value}\textsuperscript{\textit{2}} \\ 
\midrule\addlinespace[2.5pt]
count & 28 (44) & 23 (39) & 0.3 \\ 
\bottomrule
\end{tabular*}
\begin{minipage}{\linewidth}
\vspace{.05em}
\textsuperscript{\textit{1}} Mean (SD)\\
\textsuperscript{\textit{2}} Wilcoxon rank sum test\\
\end{minipage}
\end{table}

\begin{center}\rule{0.5\linewidth}{0.5pt}\end{center}

\paragraph{Step 4: OLS regression- building
intuition}\label{step-4-ols-regression--building-intuition}

\textbf{a.} Start with a simple OLS estimator of lobster counts
regressed on treatment. Use the function \texttt{summ()} from the
\href{https://jtools.jacob-long.com/}{\texttt{jtools}} package to print
the OLS output

\textbf{b.} Interpret the intercept \& predictor coefficients \emph{in
your own words}. Use full sentences and write your interpretation of the
regression results to be as clear as possible to a non-academic
audience.

There are about 23 spiny lobsters predicted, on average, in non-MPA
areas. MPA areas have, on average, about 5 more spiny lobsters than
non-MPA areas. Marine protected areas are likely associated with higher
lobster abundance; there is evidence to suggest that the influence of
MPA designation on the predicted spiny lobster count in a measured reef
is unlikely due to random chance alone (p \textless{} .05).

\begin{Shaded}
\begin{Highlighting}[]
\CommentTok{\# }\AlertTok{NOTE}\CommentTok{: We will not evaluate/interpret model fit in this assignment (e.g., R{-}square)}

\NormalTok{m1\_ols }\OtherTok{\textless{}{-}} \FunctionTok{lm}\NormalTok{(count }\SpecialCharTok{\textasciitilde{}}\NormalTok{ treat, }\AttributeTok{data =}\NormalTok{ spiny\_counts)}

\FunctionTok{summ}\NormalTok{(m1\_ols, }\AttributeTok{model.fit =} \ConstantTok{FALSE}\NormalTok{) }
\end{Highlighting}
\end{Shaded}

\begin{table}[!h]
\centering
\begin{tabular}{lr}
\toprule
\cellcolor{gray!10}{Observations} & \cellcolor{gray!10}{252}\\
Dependent variable & count\\
\cellcolor{gray!10}{Type} & \cellcolor{gray!10}{OLS linear regression}\\
\bottomrule
\end{tabular}
\end{table}  \begin{table}[!h]
\centering
\begin{threeparttable}
\begin{tabular}{lrrrr}
\toprule
  & Est. & S.E. & t val. & p\\
\midrule
\cellcolor{gray!10}{(Intercept)} & \cellcolor{gray!10}{22.73} & \cellcolor{gray!10}{3.57} & \cellcolor{gray!10}{6.36} & \cellcolor{gray!10}{0.00}\\
treat & 5.36 & 5.20 & 1.03 & 0.30\\
\bottomrule
\end{tabular}
\begin{tablenotes}
\item Standard errors: OLS
\end{tablenotes}
\end{threeparttable}
\end{table}

\textbf{c.} Check the model assumptions using the \texttt{check\_model}
function from the \texttt{performance} package

\textbf{d.} Explain the results of the 4 diagnostic plots. Why are we
getting this result?

\begin{itemize}
\item
  \textbf{qqplot}: The residuals do not fall along the line and show a
  pattern (curve), indicating that the relationship between the
  treatment group and the lobster count may not be linear.
\item
  \textbf{Normality of Residuals:} The distribution of the residuals is
  non-normal, indicating that the model may not capture underlying
  patterns between variables or potential outliers in the spiny lobster
  count.
\item
  \textbf{Homogeneity of Variance:} There is clear heteroscedasticity in
  the homogeneity of variance graph, indicating that the spread or
  variance of the spiny lobster count is not equal across the treatment
  groups (MPA or non-MPA).
\item
  \textbf{Posterior Predictive Check:} The posterior predictive check
  indicates a poor model fit, as the simulated outcomes diverge from the
  observed data. This suggests that the relationship between the
  treatment and spiny lobster count is not sufficiently captured by a
  linear regression model; another model, such as a Poisson regression
  for count data, may be a better alternative.
\end{itemize}

\begin{Shaded}
\begin{Highlighting}[]
\FunctionTok{check\_model}\NormalTok{(m1\_ols,  }\AttributeTok{check =} \StringTok{"qq"}\NormalTok{ )}
\end{Highlighting}
\end{Shaded}

\pandocbounded{\includegraphics[keepaspectratio]{hw1-lobstrs-eds241_files/figure-latex/unnamed-chunk-12-1.pdf}}

\begin{Shaded}
\begin{Highlighting}[]
\FunctionTok{check\_model}\NormalTok{(m1\_ols, }\AttributeTok{check =} \StringTok{"normality"}\NormalTok{)}
\end{Highlighting}
\end{Shaded}

\pandocbounded{\includegraphics[keepaspectratio]{hw1-lobstrs-eds241_files/figure-latex/unnamed-chunk-13-1.pdf}}

\begin{Shaded}
\begin{Highlighting}[]
\FunctionTok{check\_model}\NormalTok{(m1\_ols, }\AttributeTok{check =} \StringTok{"homogeneity"}\NormalTok{)}
\end{Highlighting}
\end{Shaded}

\pandocbounded{\includegraphics[keepaspectratio]{hw1-lobstrs-eds241_files/figure-latex/unnamed-chunk-14-1.pdf}}

\begin{Shaded}
\begin{Highlighting}[]
\FunctionTok{check\_model}\NormalTok{(m1\_ols, }\AttributeTok{check =} \StringTok{"pp\_check"}\NormalTok{)}
\end{Highlighting}
\end{Shaded}

\pandocbounded{\includegraphics[keepaspectratio]{hw1-lobstrs-eds241_files/figure-latex/unnamed-chunk-15-1.pdf}}

\begin{center}\rule{0.5\linewidth}{0.5pt}\end{center}

\paragraph{Step 5: Fitting GLMs}\label{step-5-fitting-glms}

\textbf{a.} Estimate a Poisson regression model using the \texttt{glm()}
function

\textbf{b.} Interpret the predictor coefficient in your own words. Use
full sentences and write your interpretation of the results to be as
clear as possible to a non-academic audience.

Spiny lobster counts are significantly higher in MPA designated areas
compared to non-MPA sites (p \textless{} .05). MPAs are associated with
a 24\% higher average spiny lobster count compared to non-MPA sites.

\textbf{c.} Explain the statistical concept of dispersion and
overdispersion in the context of this model.

In a Poisson regression, dispersion refers to the equivalence of the
mean and variation in the response/outcome variable, such as the spiny
lobster count in this model. With overdispersion, the variance in the
outcome variable (ie. spiny lobster counts) exceeds the mean, indicating
that the model could underestimate the standard errors from the model.
Overdispersion in this model could indicate variation in spiny lobster
counts due to under-represented or unaccounted factors, such as habitat
differences within sites.

\textbf{d.} Compare results with previous model, explain change in the
significance of the treatment effect.

Compared to the previous linear model, the treatment effect in the
Poisson model is now statistically significant. This change occurs
because the Poisson model appropriately accounts for the discrete count
data and the mean-variance relationship for counts. The linear model
previously assumed constant variance and a continuous response; by
modeling the data on the log scale with the Poisson distribution, the
model better captures the underlying distribution of spiny lobster
counts. As a result, the Poisson model reveals a significant positive
effect of treatment (MPA) on lobster abundance compared to non-MPA
transects.

\begin{Shaded}
\begin{Highlighting}[]
\CommentTok{\#HINT1: Incidence Ratio Rate (IRR): Exponentiation of beta returns coefficient which is interpreted as the \textquotesingle{}percent change\textquotesingle{} for a one unit increase in the predictor }


\CommentTok{\#HINT2: For the second glm() argument \textasciigrave{}family\textasciigrave{} use the following specification option \textasciigrave{}family = poisson(link = "log")\textasciigrave{}}

\NormalTok{m2\_pois }\OtherTok{\textless{}{-}} \FunctionTok{glm}\NormalTok{(count }\SpecialCharTok{\textasciitilde{}}\NormalTok{ treat, }\AttributeTok{data =}\NormalTok{ spiny\_counts, }\AttributeTok{family =} \FunctionTok{poisson}\NormalTok{(}\AttributeTok{link =} \StringTok{"log"}\NormalTok{))}


\FunctionTok{summ}\NormalTok{(m2\_pois) }\CommentTok{\# exp = T will return the IRR (exponentation of beta)}
\end{Highlighting}
\end{Shaded}

\begin{table}[!h]
\centering
\begin{tabular}{lr}
\toprule
\cellcolor{gray!10}{Observations} & \cellcolor{gray!10}{252}\\
Dependent variable & count\\
\cellcolor{gray!10}{Type} & \cellcolor{gray!10}{Generalized linear model}\\
Family & poisson\\
\cellcolor{gray!10}{Link} & \cellcolor{gray!10}{log}\\
\bottomrule
\end{tabular}
\end{table} \begin{table}[!h]
\centering
\begin{tabular}{lr}
\toprule
\cellcolor{gray!10}{$\chi^2$(1)} & \cellcolor{gray!10}{71.36}\\
p & 0.00\\
\cellcolor{gray!10}{Pseudo-R² (Cragg-Uhler)} & \cellcolor{gray!10}{0.25}\\
Pseudo-R² (McFadden) & 0.01\\
\cellcolor{gray!10}{AIC} & \cellcolor{gray!10}{11365.62}\\
\addlinespace
BIC & 11372.68\\
\bottomrule
\end{tabular}
\end{table} \begin{table}[!h]
\centering
\begin{threeparttable}
\begin{tabular}{lrrrr}
\toprule
  & Est. & S.E. & z val. & p\\
\midrule
\cellcolor{gray!10}{(Intercept)} & \cellcolor{gray!10}{3.12} & \cellcolor{gray!10}{0.02} & \cellcolor{gray!10}{171.74} & \cellcolor{gray!10}{0.00}\\
treat & 0.21 & 0.03 & 8.44 & 0.00\\
\bottomrule
\end{tabular}
\begin{tablenotes}
\item Standard errors: MLE
\end{tablenotes}
\end{threeparttable}
\end{table}

\begin{Shaded}
\begin{Highlighting}[]
\CommentTok{\# Calculate percent change: ((New IRR {-} Old IRR) / Old IRR) * 100, where "1" refers to the treatment group}
\FunctionTok{print}\NormalTok{(}\FunctionTok{paste}\NormalTok{(}\StringTok{"The percent change in the predictor is"}\NormalTok{, ((}\FloatTok{1.24}\DecValTok{{-}1}\NormalTok{)}\SpecialCharTok{/}\DecValTok{1}\NormalTok{)}\SpecialCharTok{*}\DecValTok{100}\NormalTok{, }\StringTok{"\%."}\NormalTok{))}
\end{Highlighting}
\end{Shaded}

\begin{verbatim}
## [1] "The percent change in the predictor is 24 %."
\end{verbatim}

\textbf{e.} Check the model assumptions. Explain results.

\begin{itemize}
\item
  \textbf{Posterior Predictive Check:} The model's predicted intervals
  slightly correspond to the observed data; however, the simulated
  intervals still do not accurately fit or capture the range of the
  observed data, indicating a poor fit likely due to overdispersion or
  zero-inflation.
\item
  \textbf{Misdirected Dispersion and Zero-Inflation:} The observed
  residual variance is higher than the predicted residual variance,
  especially at larger predicted means. As such, the model may
  underestimate the variance in the count outcome and not account for
  zero-inflation (many zeroes) or overdispersion.
\item
  \textbf{Homogeneity of Variation:} The plot shows that many residuals
  are spread unevenly compared to the reference line. As a result,
  constance variance may be violated; residuals appear to be
  heteroscedastic.
\end{itemize}

\begin{itemize}
\item
  \textbf{Influential Observation:} All points appear to be near the
  boundary of the contour with few extreme residuals; some observations
  have a higher influence which could potentially affect the parameter
  estimates in the model.
\item
  \textbf{Distribution of Quantile Residuals}: There is a clear
  deviation away from the line in the qqplot. As a result, the residuals
  are not normally distributed.
\end{itemize}

\textbf{f.} Conduct tests for over-dispersion \& zero-inflation. Explain
results.

The Poisson distribution assumes that the mean and variance are equal.
However, the variance is much larger than the mean and the p value is
very significant, indicating strong overdispersion (the model is
underestimating the variability in the data). Although the model
predicts no zeros, the data indicates 27 zeros. This indicates the model
is under fitting zeros.

\begin{Shaded}
\begin{Highlighting}[]
\FunctionTok{check\_model}\NormalTok{(m2\_pois)}
\end{Highlighting}
\end{Shaded}

\pandocbounded{\includegraphics[keepaspectratio]{hw1-lobstrs-eds241_files/figure-latex/unnamed-chunk-17-1.pdf}}

\begin{Shaded}
\begin{Highlighting}[]
\FunctionTok{check\_overdispersion}\NormalTok{(m2\_pois)}
\end{Highlighting}
\end{Shaded}

\begin{verbatim}
## # Overdispersion test
## 
##        dispersion ratio =    67.033
##   Pearson's Chi-Squared = 16758.289
##                 p-value =   < 0.001
\end{verbatim}

\begin{Shaded}
\begin{Highlighting}[]
\FunctionTok{check\_zeroinflation}\NormalTok{(m2\_pois)}
\end{Highlighting}
\end{Shaded}

\begin{verbatim}
## # Check for zero-inflation
## 
##    Observed zeros: 27
##   Predicted zeros: 0
##             Ratio: 0.00
\end{verbatim}

\textbf{g.} Fit a negative binomial model using the function glm.nb()
from the package \texttt{MASS} and check model diagnostics

\textbf{h.} In 1-2 sentences explain rationale for fitting this GLM
model.

Because the Poisson model showed zero inflation and overdispersion, a
negative binomial GLM is appropriate because it introduces an extra
dispersion parameter to account for variance exceeding the mean and can
better handle excess zeros (ie. like in the observed data), providing
more accurate estimates and inference. This approach prevents
overestimating the treatment effect's significance in the Poisson model.

\textbf{i.} Interpret the treatment estimate result in your own words.
Compare with results from the previous model.

In the Poisson model, the treatment effect was statistically significant
(p \textless{} 0.05), indicating a 24\% increase in spiny lobster counts
in transects within MPAs compared to non-MPAs. However, this model
likely overestimated significance due to zero inflation and
overdispersion. The negative binomial model also estimated a 24\%
increase, but the effect was not statistically significant (p = 0.22).
This model provides a better fit, as the posterior predictive checks
show simulations closely match the observed data and nearly all
residuals fall along the line in the qqplot. Although the negative
binomial model accounts for overdispersion, some zero inflation remains,
which may influence the observed significance of the treatment effect.

\begin{Shaded}
\begin{Highlighting}[]
\FunctionTok{library}\NormalTok{(MASS) }\DocumentationTok{\#\# }\AlertTok{NOTE}\DocumentationTok{: The \textasciigrave{}select()\textasciigrave{} function is masked. Use: \textasciigrave{}dplyr::select()\textasciigrave{} \#\#}
\end{Highlighting}
\end{Shaded}

\begin{Shaded}
\begin{Highlighting}[]
\CommentTok{\# }\AlertTok{NOTE}\CommentTok{: The \textasciigrave{}glm.nb()\textasciigrave{} function does not require a \textasciigrave{}family\textasciigrave{} argument}

\NormalTok{m3\_nb }\OtherTok{\textless{}{-}} \FunctionTok{glm.nb}\NormalTok{(count }\SpecialCharTok{\textasciitilde{}}\NormalTok{ treat, }\AttributeTok{data =}\NormalTok{ spiny\_counts) }

\FunctionTok{summ}\NormalTok{(m3\_nb)}
\end{Highlighting}
\end{Shaded}

\begin{table}[!h]
\centering
\begin{tabular}{lr}
\toprule
\cellcolor{gray!10}{Observations} & \cellcolor{gray!10}{252}\\
Dependent variable & count\\
\cellcolor{gray!10}{Type} & \cellcolor{gray!10}{Generalized linear model}\\
Family & Negative Binomial(0.55)\\
\cellcolor{gray!10}{Link} & \cellcolor{gray!10}{log}\\
\bottomrule
\end{tabular}
\end{table} \begin{table}[!h]
\centering
\begin{tabular}{lr}
\toprule
\cellcolor{gray!10}{$\chi^2$(250)} & \cellcolor{gray!10}{1.52}\\
p & 0.22\\
\cellcolor{gray!10}{Pseudo-R² (Cragg-Uhler)} & \cellcolor{gray!10}{0.01}\\
Pseudo-R² (McFadden) & 0.00\\
\cellcolor{gray!10}{AIC} & \cellcolor{gray!10}{2088.53}\\
\addlinespace
BIC & 2099.12\\
\bottomrule
\end{tabular}
\end{table} \begin{table}[!h]
\centering
\begin{threeparttable}
\begin{tabular}{lrrrr}
\toprule
  & Est. & S.E. & z val. & p\\
\midrule
\cellcolor{gray!10}{(Intercept)} & \cellcolor{gray!10}{3.12} & \cellcolor{gray!10}{0.12} & \cellcolor{gray!10}{26.40} & \cellcolor{gray!10}{0.00}\\
treat & 0.21 & 0.17 & 1.23 & 0.22\\
\bottomrule
\end{tabular}
\begin{tablenotes}
\item Standard errors: MLE
\end{tablenotes}
\end{threeparttable}
\end{table}

\begin{Shaded}
\begin{Highlighting}[]
\CommentTok{\# Calculate percent change: ((New IRR {-} Old IRR) / Old IRR) * 100, where "1" refers to the treatment group}
\FunctionTok{print}\NormalTok{(}\FunctionTok{paste}\NormalTok{(}\StringTok{"The percent change in the predictor is"}\NormalTok{, ((}\FloatTok{1.24}\DecValTok{{-}1}\NormalTok{)}\SpecialCharTok{/}\DecValTok{1}\NormalTok{)}\SpecialCharTok{*}\DecValTok{100}\NormalTok{, }\StringTok{"\%."}\NormalTok{))}
\end{Highlighting}
\end{Shaded}

\begin{verbatim}
## [1] "The percent change in the predictor is 24 %."
\end{verbatim}

\begin{Shaded}
\begin{Highlighting}[]
\FunctionTok{check\_overdispersion}\NormalTok{(m3\_nb)}
\end{Highlighting}
\end{Shaded}

\begin{verbatim}
## # Overdispersion test
## 
##  dispersion ratio = 1.398
##           p-value = 0.088
\end{verbatim}

\begin{Shaded}
\begin{Highlighting}[]
\FunctionTok{check\_zeroinflation}\NormalTok{(m3\_nb)}
\end{Highlighting}
\end{Shaded}

\begin{verbatim}
## # Check for zero-inflation
## 
##    Observed zeros: 27
##   Predicted zeros: 30
##             Ratio: 1.12
\end{verbatim}

\begin{Shaded}
\begin{Highlighting}[]
\FunctionTok{check\_predictions}\NormalTok{(m3\_nb)}
\end{Highlighting}
\end{Shaded}

\pandocbounded{\includegraphics[keepaspectratio]{hw1-lobstrs-eds241_files/figure-latex/unnamed-chunk-24-1.pdf}}

\begin{Shaded}
\begin{Highlighting}[]
\FunctionTok{check\_model}\NormalTok{(m3\_nb)}
\end{Highlighting}
\end{Shaded}

\pandocbounded{\includegraphics[keepaspectratio]{hw1-lobstrs-eds241_files/figure-latex/unnamed-chunk-25-1.pdf}}

\begin{center}\rule{0.5\linewidth}{0.5pt}\end{center}

\paragraph{Step 6: Compare models}\label{step-6-compare-models}

\textbf{a.} Use the \texttt{export\_summ()} function from the
\texttt{jtools} package to look at the three regression models you fit
side-by-side.

\textbf{c.} Write a short paragraph comparing the results. Is the
treatment effect \texttt{robust} or stable across the model
specifications.

Across all three models, the estimated treatment effect remains rather
consistent (stable), corresponding to a roughly 24\% increase in spiny
lobster counts within MPAs compared to non-MPAs based on the calculated
percent change of the treatment effect in each model. While the Poisson
model indicates a statistically significant effect, the OLS and negative
binomial models show that the effect is not significant even if
overdispersion is accounted for (the negative binomial model still
experiences zero-inflation). This indicates that the size of the
treatment effect appears consistent across models, but whether it
appears statistically significant depends on which model is used and how
the data are distributed.

\begin{Shaded}
\begin{Highlighting}[]
\FunctionTok{export\_summs}\NormalTok{( }\CommentTok{\# ADD MODELS}
\NormalTok{        m1\_ols, m2\_pois, m3\_nb,}
             \AttributeTok{model.names =} \FunctionTok{c}\NormalTok{(}\StringTok{"OLS"}\NormalTok{,}\StringTok{"Poisson"}\NormalTok{, }\StringTok{"NB"}\NormalTok{),}
             \AttributeTok{statistics =} \StringTok{"none"}\NormalTok{)}
\end{Highlighting}
\end{Shaded}

 
  \providecommand{\huxb}[2]{\arrayrulecolor[RGB]{#1}\global\arrayrulewidth=#2pt}
  \providecommand{\huxvb}[2]{\color[RGB]{#1}\vrule width #2pt}
  \providecommand{\huxtpad}[1]{\rule{0pt}{#1}}
  \providecommand{\huxbpad}[1]{\rule[-#1]{0pt}{#1}}

\begin{table}[ht]
\begin{centerbox}
\begin{threeparttable}
 \setlength{\tabcolsep}{0pt}
\begin{tabular}{l l l l}


\hhline{>{\huxb{0, 0, 0}{0.8}}->{\huxb{0, 0, 0}{0.8}}->{\huxb{0, 0, 0}{0.8}}->{\huxb{0, 0, 0}{0.8}}-}
\arrayrulecolor{black}

\multicolumn{1}{!{\huxvb{0, 0, 0}{0}}c!{\huxvb{0, 0, 0}{0}}}{\huxtpad{6pt + 1em}\centering \hspace{6pt}  \hspace{6pt}\huxbpad{6pt}} &
\multicolumn{1}{c!{\huxvb{0, 0, 0}{0}}}{\huxtpad{6pt + 1em}\centering \hspace{6pt} OLS \hspace{6pt}\huxbpad{6pt}} &
\multicolumn{1}{c!{\huxvb{0, 0, 0}{0}}}{\huxtpad{6pt + 1em}\centering \hspace{6pt} Poisson \hspace{6pt}\huxbpad{6pt}} &
\multicolumn{1}{c!{\huxvb{0, 0, 0}{0}}}{\huxtpad{6pt + 1em}\centering \hspace{6pt} NB \hspace{6pt}\huxbpad{6pt}} \tabularnewline[-0.5pt]


\hhline{>{\huxb{255, 255, 255}{0.4}}->{\huxb{0, 0, 0}{0.4}}->{\huxb{0, 0, 0}{0.4}}->{\huxb{0, 0, 0}{0.4}}-}
\arrayrulecolor{black}

\multicolumn{1}{!{\huxvb{0, 0, 0}{0}}l!{\huxvb{0, 0, 0}{0}}}{\huxtpad{6pt + 1em}\raggedright \hspace{6pt} (Intercept) \hspace{6pt}\huxbpad{6pt}} &
\multicolumn{1}{r!{\huxvb{0, 0, 0}{0}}}{\huxtpad{6pt + 1em}\raggedleft \hspace{6pt} 22.73 *** \hspace{6pt}\huxbpad{6pt}} &
\multicolumn{1}{r!{\huxvb{0, 0, 0}{0}}}{\huxtpad{6pt + 1em}\raggedleft \hspace{6pt} 3.12 *** \hspace{6pt}\huxbpad{6pt}} &
\multicolumn{1}{r!{\huxvb{0, 0, 0}{0}}}{\huxtpad{6pt + 1em}\raggedleft \hspace{6pt} 3.12 *** \hspace{6pt}\huxbpad{6pt}} \tabularnewline[-0.5pt]


\hhline{}
\arrayrulecolor{black}

\multicolumn{1}{!{\huxvb{0, 0, 0}{0}}l!{\huxvb{0, 0, 0}{0}}}{\huxtpad{6pt + 1em}\raggedright \hspace{6pt}  \hspace{6pt}\huxbpad{6pt}} &
\multicolumn{1}{r!{\huxvb{0, 0, 0}{0}}}{\huxtpad{6pt + 1em}\raggedleft \hspace{6pt} (3.57)\hphantom{0}\hphantom{0}\hphantom{0} \hspace{6pt}\huxbpad{6pt}} &
\multicolumn{1}{r!{\huxvb{0, 0, 0}{0}}}{\huxtpad{6pt + 1em}\raggedleft \hspace{6pt} (0.02)\hphantom{0}\hphantom{0}\hphantom{0} \hspace{6pt}\huxbpad{6pt}} &
\multicolumn{1}{r!{\huxvb{0, 0, 0}{0}}}{\huxtpad{6pt + 1em}\raggedleft \hspace{6pt} (0.12)\hphantom{0}\hphantom{0}\hphantom{0} \hspace{6pt}\huxbpad{6pt}} \tabularnewline[-0.5pt]


\hhline{}
\arrayrulecolor{black}

\multicolumn{1}{!{\huxvb{0, 0, 0}{0}}l!{\huxvb{0, 0, 0}{0}}}{\huxtpad{6pt + 1em}\raggedright \hspace{6pt} treat \hspace{6pt}\huxbpad{6pt}} &
\multicolumn{1}{r!{\huxvb{0, 0, 0}{0}}}{\huxtpad{6pt + 1em}\raggedleft \hspace{6pt} 5.36\hphantom{0}\hphantom{0}\hphantom{0}\hphantom{0} \hspace{6pt}\huxbpad{6pt}} &
\multicolumn{1}{r!{\huxvb{0, 0, 0}{0}}}{\huxtpad{6pt + 1em}\raggedleft \hspace{6pt} 0.21 *** \hspace{6pt}\huxbpad{6pt}} &
\multicolumn{1}{r!{\huxvb{0, 0, 0}{0}}}{\huxtpad{6pt + 1em}\raggedleft \hspace{6pt} 0.21\hphantom{0}\hphantom{0}\hphantom{0}\hphantom{0} \hspace{6pt}\huxbpad{6pt}} \tabularnewline[-0.5pt]


\hhline{}
\arrayrulecolor{black}

\multicolumn{1}{!{\huxvb{0, 0, 0}{0}}l!{\huxvb{0, 0, 0}{0}}}{\huxtpad{6pt + 1em}\raggedright \hspace{6pt}  \hspace{6pt}\huxbpad{6pt}} &
\multicolumn{1}{r!{\huxvb{0, 0, 0}{0}}}{\huxtpad{6pt + 1em}\raggedleft \hspace{6pt} (5.20)\hphantom{0}\hphantom{0}\hphantom{0} \hspace{6pt}\huxbpad{6pt}} &
\multicolumn{1}{r!{\huxvb{0, 0, 0}{0}}}{\huxtpad{6pt + 1em}\raggedleft \hspace{6pt} (0.03)\hphantom{0}\hphantom{0}\hphantom{0} \hspace{6pt}\huxbpad{6pt}} &
\multicolumn{1}{r!{\huxvb{0, 0, 0}{0}}}{\huxtpad{6pt + 1em}\raggedleft \hspace{6pt} (0.17)\hphantom{0}\hphantom{0}\hphantom{0} \hspace{6pt}\huxbpad{6pt}} \tabularnewline[-0.5pt]


\hhline{>{\huxb{0, 0, 0}{0.8}}->{\huxb{0, 0, 0}{0.8}}->{\huxb{0, 0, 0}{0.8}}->{\huxb{0, 0, 0}{0.8}}-}
\arrayrulecolor{black}

\multicolumn{4}{!{\huxvb{0, 0, 0}{0}}l!{\huxvb{0, 0, 0}{0}}}{\huxtpad{6pt + 1em}\raggedright \hspace{6pt} *** p $<$ 0.001; ** p $<$ 0.01; * p $<$ 0.05. \hspace{6pt}\huxbpad{6pt}} \tabularnewline[-0.5pt]


\hhline{}
\arrayrulecolor{black}
\end{tabular}
\end{threeparttable}\par\end{centerbox}

\end{table}
 

\begin{Shaded}
\begin{Highlighting}[]
\CommentTok{\# Calculate percent change}
\NormalTok{(}\FloatTok{5.36}\SpecialCharTok{/}\FloatTok{22.73}\NormalTok{) }\SpecialCharTok{*}\DecValTok{100}  \CommentTok{\# Linear regression}
\end{Highlighting}
\end{Shaded}

\begin{verbatim}
## [1] 23.58117
\end{verbatim}

\begin{Shaded}
\begin{Highlighting}[]
\NormalTok{(}\FunctionTok{exp}\NormalTok{(.}\DecValTok{21}\NormalTok{)}\SpecialCharTok{{-}}\DecValTok{1}\SpecialCharTok{/}\DecValTok{1}\NormalTok{)}\SpecialCharTok{*}\DecValTok{100} \CommentTok{\# Poisson regression}
\end{Highlighting}
\end{Shaded}

\begin{verbatim}
## [1] 23.36781
\end{verbatim}

\begin{Shaded}
\begin{Highlighting}[]
\NormalTok{(}\FunctionTok{exp}\NormalTok{(.}\DecValTok{21}\NormalTok{)}\SpecialCharTok{{-}}\DecValTok{1}\SpecialCharTok{/}\DecValTok{1}\NormalTok{)}\SpecialCharTok{*}\DecValTok{100} \CommentTok{\# Negative binomial regression}
\end{Highlighting}
\end{Shaded}

\begin{verbatim}
## [1] 23.36781
\end{verbatim}

\begin{center}\rule{0.5\linewidth}{0.5pt}\end{center}

\paragraph{Step 7: Building intuition - fixed
effects}\label{step-7-building-intuition---fixed-effects}

\textbf{a.} Create new \texttt{df} with the \texttt{year} variable
converted to a factor

\textbf{b.} Run the following negative binomial model using
\texttt{glm.nb()}

\begin{itemize}
\tightlist
\item
  Add fixed effects for \texttt{year} (i.e., dummy coefficients)
\item
  Include an interaction term between variables \texttt{treat} \&
  \texttt{year} (\texttt{treat*year})
\end{itemize}

\textbf{c.} Take a look at the regression output. Each coefficient
provides a comparison or the difference in means for a specific
sub-group in the data. Informally, describe the what the model has
estimated at a conceptual level (NOTE: you do not have to interpret
coefficients individually)

The negative binomial model is estimating how spiny lobster counts vary
across years, between treatment groups, and in the interaction between
year and treatment. Essentially, it is comparing the expected counts for
different sub groups (ex. MPA transects vs non-MPA transects each year)
while also accounting for overdispersion in the outcome. The model
captures both overall trends across years and how the effect of
treatment changes over time.

\textbf{d.} Explain why the main effect for treatment is negative? *Does
this result make sense?

The main effect for treatment is negative because it represents the
baseline difference between the treatment (transects in an MPA) and
control group (transects in non-MPAs) when all years are set to zero
(ie. the reference year). This does not mean that the treatment reduces
the spiny lobster counts overall but indicates that in the reference
year, the MPA transects had a lower average count of spiny lobsters than
non-MPA transects. The positive ``treatment by year'' interaction terms
show that counts in MPAs increase over time, indicating consistency with
expected benefits of protection provided by MPAs.

\begin{Shaded}
\begin{Highlighting}[]
\NormalTok{ff\_counts }\OtherTok{\textless{}{-}}\NormalTok{ spiny\_counts }\SpecialCharTok{\%\textgreater{}\%} 
    \FunctionTok{mutate}\NormalTok{(}\AttributeTok{year=}\FunctionTok{as\_factor}\NormalTok{(year))}
    
\NormalTok{m5\_fixedeffs }\OtherTok{\textless{}{-}} \FunctionTok{glm.nb}\NormalTok{(}
\NormalTok{    count }\SpecialCharTok{\textasciitilde{}} 
\NormalTok{        treat }\SpecialCharTok{+}
\NormalTok{        year }\SpecialCharTok{+}
\NormalTok{        treat}\SpecialCharTok{*}\NormalTok{year,}
    \AttributeTok{data =}\NormalTok{ ff\_counts)}

\FunctionTok{summ}\NormalTok{(m5\_fixedeffs, }\AttributeTok{model.fit =} \ConstantTok{FALSE}\NormalTok{)}
\end{Highlighting}
\end{Shaded}

\begin{table}[!h]
\centering
\begin{tabular}{lr}
\toprule
\cellcolor{gray!10}{Observations} & \cellcolor{gray!10}{252}\\
Dependent variable & count\\
\cellcolor{gray!10}{Type} & \cellcolor{gray!10}{Generalized linear model}\\
Family & Negative Binomial(0.8129)\\
\cellcolor{gray!10}{Link} & \cellcolor{gray!10}{log}\\
\bottomrule
\end{tabular}
\end{table}  \begin{table}[!h]
\centering
\begin{threeparttable}
\begin{tabular}{lrrrr}
\toprule
  & Est. & S.E. & z val. & p\\
\midrule
\cellcolor{gray!10}{(Intercept)} & \cellcolor{gray!10}{2.35} & \cellcolor{gray!10}{0.26} & \cellcolor{gray!10}{8.89} & \cellcolor{gray!10}{0.00}\\
treat & -1.72 & 0.42 & -4.12 & 0.00\\
\cellcolor{gray!10}{year2013} & \cellcolor{gray!10}{-0.35} & \cellcolor{gray!10}{0.38} & \cellcolor{gray!10}{-0.93} & \cellcolor{gray!10}{0.35}\\
year2014 & 0.08 & 0.37 & 0.21 & 0.84\\
\cellcolor{gray!10}{year2015} & \cellcolor{gray!10}{0.86} & \cellcolor{gray!10}{0.37} & \cellcolor{gray!10}{2.32} & \cellcolor{gray!10}{0.02}\\
\addlinespace
year2016 & 0.90 & 0.37 & 2.43 & 0.01\\
\cellcolor{gray!10}{year2017} & \cellcolor{gray!10}{1.56} & \cellcolor{gray!10}{0.37} & \cellcolor{gray!10}{4.25} & \cellcolor{gray!10}{0.00}\\
year2018 & 1.04 & 0.37 & 2.81 & 0.00\\
\cellcolor{gray!10}{treat:year2013} & \cellcolor{gray!10}{1.52} & \cellcolor{gray!10}{0.57} & \cellcolor{gray!10}{2.66} & \cellcolor{gray!10}{0.01}\\
treat:year2014 & 2.14 & 0.56 & 3.80 & 0.00\\
\addlinespace
\cellcolor{gray!10}{treat:year2015} & \cellcolor{gray!10}{2.12} & \cellcolor{gray!10}{0.56} & \cellcolor{gray!10}{3.79} & \cellcolor{gray!10}{0.00}\\
treat:year2016 & 1.40 & 0.56 & 2.50 & 0.01\\
\cellcolor{gray!10}{treat:year2017} & \cellcolor{gray!10}{1.55} & \cellcolor{gray!10}{0.56} & \cellcolor{gray!10}{2.77} & \cellcolor{gray!10}{0.01}\\
treat:year2018 & 2.62 & 0.56 & 4.69 & 0.00\\
\bottomrule
\end{tabular}
\begin{tablenotes}
\item Standard errors: MLE
\end{tablenotes}
\end{threeparttable}
\end{table}

\textbf{e.} Look at the model predictions: Use the
\texttt{interact\_plot()} function from package \texttt{interactions} to
plot mean predictions by year and treatment status.

\textbf{f.} Re-evaluate your responses (c) and (d) above.

\begin{Shaded}
\begin{Highlighting}[]
\FunctionTok{interact\_plot}\NormalTok{(m5\_fixedeffs, }\AttributeTok{pred =}\NormalTok{ year, }\AttributeTok{modx =}\NormalTok{ treat, }
              \AttributeTok{outcome.scale =} \StringTok{"link"}\NormalTok{) }\CommentTok{\# }\AlertTok{NOTE}\CommentTok{: y{-}axis on log{-}scale}
\end{Highlighting}
\end{Shaded}

\pandocbounded{\includegraphics[keepaspectratio]{hw1-lobstrs-eds241_files/figure-latex/unnamed-chunk-29-1.pdf}}

\begin{Shaded}
\begin{Highlighting}[]
\CommentTok{\# HINT: Change \textasciigrave{}outcome.scale\textasciigrave{} to "response" to convert y{-}axis scale to counts}
\FunctionTok{interact\_plot}\NormalTok{(m5\_fixedeffs, }\AttributeTok{pred =}\NormalTok{ year, }\AttributeTok{modx =}\NormalTok{ treat, }
              \AttributeTok{outcome.scale =} \StringTok{"response"}\NormalTok{) }
\end{Highlighting}
\end{Shaded}

\pandocbounded{\includegraphics[keepaspectratio]{hw1-lobstrs-eds241_files/figure-latex/unnamed-chunk-29-2.pdf}}

\textbf{g.} Using \texttt{ggplot()} create a plot in same style as the
previous \texttt{interaction\ plot}, but displaying the original scale
of the outcome variable (lobster counts). This type of plot is commonly
used to show how the treatment effect changes across discrete time
points (i.e., panel data).

The plot should have\ldots{} - \texttt{year} on the x-axis -
\texttt{counts} on the y-axis - \texttt{mpa} as the grouping variable

\begin{Shaded}
\begin{Highlighting}[]
\CommentTok{\# Hint 1: Group counts by \textasciigrave{}year\textasciigrave{} and \textasciigrave{}mpa\textasciigrave{} and calculate the \textasciigrave{}mean\_count\textasciigrave{}}
\CommentTok{\# Hint 2: Convert variable \textasciigrave{}year\textasciigrave{} to a factor}

\NormalTok{plot\_counts }\OtherTok{\textless{}{-}}\NormalTok{ spiny\_counts }\SpecialCharTok{\%\textgreater{}\%}
  \FunctionTok{group\_by}\NormalTok{(year, mpa) }\SpecialCharTok{\%\textgreater{}\%}
  \FunctionTok{summarize}\NormalTok{(}\AttributeTok{mean\_count =} \FunctionTok{mean}\NormalTok{(count, }\AttributeTok{na.rm =} \ConstantTok{TRUE}\NormalTok{), }\AttributeTok{.groups =} \StringTok{"drop"}\NormalTok{) }\SpecialCharTok{\%\textgreater{}\%}
  \FunctionTok{ggplot}\NormalTok{(}\FunctionTok{aes}\NormalTok{(}\AttributeTok{x =} \FunctionTok{factor}\NormalTok{(year), }\AttributeTok{y =}\NormalTok{ mean\_count, }\AttributeTok{color =} \FunctionTok{factor}\NormalTok{(mpa), }\AttributeTok{group =}\NormalTok{ mpa)) }\SpecialCharTok{+}
  \FunctionTok{geom\_point}\NormalTok{(}\AttributeTok{size =} \DecValTok{3}\NormalTok{) }\SpecialCharTok{+}
  \FunctionTok{geom\_line}\NormalTok{(}\AttributeTok{size =} \DecValTok{1}\NormalTok{) }\SpecialCharTok{+}
  \FunctionTok{theme\_bw}\NormalTok{() }\SpecialCharTok{+}
  \FunctionTok{labs}\NormalTok{(}
    \AttributeTok{x =} \StringTok{"Year"}\NormalTok{,}
    \AttributeTok{y =} \StringTok{"Mean Lobster Count"}\NormalTok{,}
    \AttributeTok{color =} \StringTok{"MPA Status"}\NormalTok{, }
    \AttributeTok{title =} \StringTok{"Average Spiny Lobster Counts Overtime in MPA and Non{-}MPA transects"}
\NormalTok{  ) }\SpecialCharTok{+}
  \FunctionTok{scale\_color\_manual}\NormalTok{(}\AttributeTok{values =} \FunctionTok{c}\NormalTok{(}\StringTok{"blue"}\NormalTok{, }\StringTok{"lightblue"}\NormalTok{))}

\NormalTok{plot\_counts}
\end{Highlighting}
\end{Shaded}

\pandocbounded{\includegraphics[keepaspectratio]{hw1-lobstrs-eds241_files/figure-latex/unnamed-chunk-30-1.pdf}}

\begin{center}\rule{0.5\linewidth}{0.5pt}\end{center}

\paragraph{Step 8: Reconsider causal identification
assumptions}\label{step-8-reconsider-causal-identification-assumptions}

\begin{enumerate}
\def\labelenumi{\alph{enumi}.}
\item
  Discuss whether you think \texttt{spillover\ effects} are likely in
  this research context (see Glossary of terms;
  \url{https://docs.google.com/document/d/1RIudsVcYhWGpqC-Uftk9UTz3PIq6stVyEpT44EPNgpE/edit?usp=sharing})

  Spillover effects are possible in this study because lobsters in MPAs
  (NAPL, IVEE) could potentially migrate to nearby non-MPA reefs (AQUE,
  CARP, MOHK). This movement could violate the SUTVA principle because
  the designation of one reef as an MPA could influence outcomes at
  nearby control (non-MPA) sites. If spillover occurs, then it would
  likely reduce the observed difference between the MPAs and non-MPAs
  (control sites), making it difficult to detect a causal effect of
  protection. While this effect may be ecologically desirable due to
  potentially increases lobster counts outside of MPAs, it would be
  difficult to precisely measure the treatment effect of MPA designation
  on spiny lobster counts from a research perspective.
\item
  Explain why spillover is an issue for the identification of causal
  effects.

  Spillover is an issue for identifying causal effects because it
  violates the SUTVA assumption. The SUTVA assumption assumes that each
  unit's outcome depends on only its own treatment. Spillover could
  occur if lobsters move from MPAs to nearby non-MPAs. As such,
  spillover can potentially mask longer term treatment effects because
  the difference in effect between MPA and non-MPA is reduced, making it
  difficult to isolate for the effect of protection in MPAs on average
  spiny lobster count.
\item
  How does spillover relate to impact in this research setting?

  If the SUTVA assumption is violated by movement of lobsters between
  MPAs and non-MPAs, then it becomes more difficult to measure the
  causal effects since the difference between the MPAs and non-MPAs
  effect is reduced. Essentially, lobsters moving from MPAs into nearby
  non-MPAs could increase lobster abundance and decrease the observable
  treatment-control difference that conservationists and researchers
  rely on for measuring the impact of protection on increase species
  density.
\item
  Discuss the following causal inference assumptions in the context of
  the MPA treatment effect estimator. Evaluate if each of the assumption
  are reasonable:

  \begin{enumerate}
  \def\labelenumii{\arabic{enumii})}
  \tightlist
  \item
    SUTVA: Stable Unit Treatment Value assumption: This assumption
    assumes that each unit's outcome depends only on its own treatment
    and not on the treatment assigned to other units. SUTVA could be
    violated due to spillover effects. For instance, lobsters may move
    between an MPA and non-MPA and populate outside MPAs during their
    reproductive stage to explore varied food items. As a result, the
    non-MPA lobster counts are partially affected by neighboring MPA
    sites, reducing the difference between the treatment (MPA) and
    control (non-MPA) outcomes. So, this assumption if likely
    unreasonable unless short-term or isolated studies can potentially
    accommodate the SUTVA assumption if spillover is minimal.
  \item
    Exogeneity assumption: Based on the exogeneity assumption, the MPA
    designation should be unrelated to unobserved (environmental or
    anthropogenic) factors; the MPA designation status should not be
    correlated with the regression term. The assumption could be
    violated if MPAs are placed in areas with characteristics that also
    influence lobster outcomes, such as greater habitat quality or lower
    historical fishing pressure. These factors could independently
    affect lobster abundance and confound the estimated treatment
    effect. This assumption is probably not reasonable as MPAs are
    likely to not be randomly assigned; they may be assigned based on
    specific environmental criteria instead.
  \end{enumerate}
\end{enumerate}

\begin{center}\rule{0.5\linewidth}{0.5pt}\end{center}

\section{EXTRA CREDIT}\label{extra-credit}

\begin{quote}
Use the recent lobster abundance data with observations collected up
until 2024 (\texttt{extracredit\_sblobstrs24.csv}) to run an analysis
evaluating the effect of MPA status on lobster counts using the same
focal variables.
\end{quote}

\begin{enumerate}
\def\labelenumi{\alph{enumi}.}
\tightlist
\item
  Create a new script for the analysis on the updated data
\item
  Run at least 3 regression models \& assess model diagnostics
\item
  Compare and contrast results with the analysis from the 2012-2018 data
  sample (\textasciitilde{} 2 paragraphs)
\end{enumerate}

\begin{center}\rule{0.5\linewidth}{0.5pt}\end{center}

\pandocbounded{\includegraphics[keepaspectratio]{figures/spiny1.png}}

\end{document}
